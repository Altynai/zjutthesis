% !Mode:: "TeX:UTF-8"

\documentclass[literaturereview]{zjutreport}

\graphicspath{{figures/}}  % 定义所有的eps文件在 figures 子目录下
\begin{document}           % 开始全文

%论文题目:{中文}
\zjuttitle{基于内存数据库的大数据应用系统的设计与实现}
%作者:{中文姓名}{学号}
\zjutauthor{陈佳鹏}{200926630503}
%指导教师:{导师中文名}
\zjutmentor{陈~~~~波}
%个人信息:{毕业年份}{专业名称}
\zjutinfo{2013}{软件工程}
%学院信息:{学院中文}
\zjutcollege{计算机科学与技术学院}
%日期:{提交日期}
\zjutdate{2013年06月}

\input{preface/cover}      % 封面

\frontmatter

\pagenumbering{Roman}
\begingroup % 在组内的chapter不换行
\let\clearpage\relax % chapter之后不换页

%%%%%%%%%% 标题 %%%%%%%%%%
\titleformat{\chapter}[block]{\sihao\heiti\filcenter\bfseries}{\CJKnumber{\thechapter}}{1ex}{}{} % 标题居中,黑体三号
\chapter*{基于内存数据库的大数据应用系统的设计与实现}
\titleformat{\chapter}[block]{\xiaosi\heiti}{\CJKnumber{\thechapter}、}{1ex}{}{} % 恢复标题居左,黑体四号

%%%%%%%%%% 摘要 %%%%%%%%%%
% !Mode:: "TeX:UTF-8"

%%%%% 摘要 %%%%%
\abstractc{本文是基于内存数据库的大数据应用系统的设计与实现的一篇文献综述,先介绍项目的由来及其研究意思,然后介绍项目的国内外研究现状及难点以定位项目开发的一个大环境,明确当前同类项目的研究情况。接着本文简述内存数据库系统的结构,紧接着介绍系统开发中所需的关键技术。}

%%%%% 关键词 %%%%%
\keywordsc{内存数据库,索引结构,并发控制,T树,数据恢复,影子内存,混合日志}
     % 中文摘要

%%%%%%%%%% 正文 %%%%%%%%%%
\mainmatter
% \makeatletter
\chapter{引言}
在信息化建设中用于对数据进行维护和管理的数据库扮演了一个十分重要的角色,一个合适的数据库系统往往是信息化改造成功与否的关键所在。传统的磁盘数据库(Disk Resident Database,DRDB)由于经过几十年的发展其功能完备,稳定性好,因而经常应用于电信,银行等公司进行客户资料管理,认识部门档案管理等日常应用方面,并且一直以来有着令人满意的表现。然而伴随着科技的发展,涌现出一批新兴行业或传统行业需要新的应用,例如工业控制,数据通信,证券交易,电力调度,航空航天等。这些新的产业所要求的数据库系统通常并不需要数据库具有强大而完备的功能和进行非常复杂的事物处理的能力,却需要数据库能在指定的时刻或时间内对大量的数据进行采集,处理并能正确的响应的高速的性能。
由于“I/O瓶颈”问题,基于磁盘的数
据库系统(Oracle,SQL Server等)不能满足现代应用对数据库的实耐性处理
的要求。内存数据与磁盘数据在访问时间上相差5个数量
级,内存的速度具有嚷显的优势。内存容量增大,而价格
却在不断地下降。从20世纪80年代开始,数据库研究人
员考虑把整个或者大部分数据库放在内存中。内存数据库
(MMDB)是实时数据库研究的基础,井成为了研究的热点。

\chapter{研究意义}
内存数据库因为其快速的数据访问能力,使其能比磁盘数据库(DRDB)更适合于需要快速响应
和高事务吞吐量的应用环境。对于那些需要在严格要求的时间段内完成事务请求
的实时应用系统,和需要支持大数据量并发访问的高性能事务处理平台来讲,内
存数据库都是一个理想的选择。此外,在实际生产中,常常出现不能互相访问其内置实时数据库的信息,从而使大量
信息冗余重复存在于各系统中,也就是出现数据孤岛。为了解决这个问题,必须对实时数
据库的数据管理进行合理规划以建立开放的实时数据库系统,使之能够提供高速、及时的
实时数据服务。

\chapter{国内外研究现状}
内存数据库的研究始于20世纪80年代,并逐渐吸引了越来越多的研究者的研兴趣。经过20多年的发展,时至今日研究者们已对它的体系结构,数据组织与存取方法,事务处理,并发控制和恢复备份技术进行了大量的探讨和研究,针对硬件,软件和算法设计提出了许多不同的策略和实现方案,取得了丰硕的成果。目前内存数据库的研究主要集中在一下几个方向:

1. 内存数据库的体系结构,内存数据库的体系结构包括系统的习题结构和存储体系结构两部分内容,系统体系结构方面主要侧重于研究多处理器在MMDB中的应用,而存储体系结构方面则侧重于研究非易失性内存(NV-RAM,UPS RAM等)在MMDB中的应用。

2. 事务处理,事务的处理主要对事务的提交及与之相联的日志记录、查询优化,尤其是联机查询的优化进行了多方面的研究,开发了一些有效的技术。如开发了“提前提交”等策略以优化事务的初始如对事务的初始处理、提交处理、并发控制、完整性和安全性检验等,加速并发事务的响应时间,查询的优化主要在于对减少查询中中间关系的算法的研究。

3. 数据组织与存取方法,这方面主要针对MMDB存储介质的特性设计了许多适合内存特性的数据存储组织结构和存取方式,索引结构和存储策略,MMDB的压缩,如AVL树索引结构,区段式组织结构,位图分配法等。

4. MMDB的备份和恢复,备份方面更是提出了多种方案,为了预防MMDB的崩溃主要是结合检查点和日志来保证MMDB(MMDB)在崩溃后的可恢复性,提出了许多具体的算法如FUZZY(模糊)检验点策略,BLACK/WHITE(黑/白)策略,COPY-ON-UPDATE(变更拷贝)检验点等,同时研究了非易失性内存在MMDB备份中的应用,如“影子内存”技术等。而备份恢复方面则是对MMDB重启之后的装入策略的研究,结合内存数据中数据使用的频率和优先级提出一种最优的装入策略,目前提出了“部分重装”等策略,MMDB的目录恢复后就启动系统,然后根据要求再继续重装,从而提高CPU的使用率和事务的吞吐量。

5. MMDB的并发控制,在短小事务时,往往采用大力度的锁,MMDB近似于串行处理,串行化执行事务一方面使得并发控制的代价几乎完全消除,同时也能减少CPU的cache缓存和虚拟内存页表TLB的刷新频率;而对于长事务和多处理机环境串行处理明显不合适,这样就必需应用而合适的锁机制来支持并发操作,目前提出了二级层次封锁协议方案、乐观并发控制方法、使用可扩展的哈希技术的方法等。

近些年来随着大容量廉价的内存投入市场,使得上诉的各种技术也逐渐在内存数据库的设计和实现上得到了应用,内存数据库也不在停留在理论研究阶段,内存数据库走向了实际应用阶段,各高校和研究机构发布了研究模型系统,一些公司也推出了用于工业的商用内存数据库系统。目前工业应用上比较流行的商用系统有Oracle公司的内存数据库Berkeley DB,内存数据库SQLite,开源的内存数据库FastDB,以及McObject公司的eXtremDB。他们的特点如表~\ref{tab:table1}~所示: 

\begin{table}[htbp]
\caption{主流内存数据库}\label{tab:table1}
\vspace{0.5em}{\wuhao
\begin{tabularx}{\textwidth}{llX}
\toprule[1.5pt]
内存数据库名 & 厂商 & 特点\\
\midrule[1pt]
Berkeley DB & Oracle & Berkeley DB数据库系统简单、小巧、可靠、高性能,提供了一系列应用程序接口(API),应用程序和Berkeley DB所提供的库在一起编译成为可执行程序。每一个记录由关键字和数据(KEY/VALUE)组成的键值对构成。\\
SQLite & 开源 & SQLite是一款轻型的数据库,它的设计目标是针对嵌入式系统,提供了很多语言的接口,支持SQL语句,支持事务处理,它占用资源非常的低,支持跨平台操作,操作使用简单。每完成一次操作需要进行内存和磁盘之间的同步。\\
ExtremeDB & McObject & ExtremDB是为实时系统及嵌入式系统而特别设计的,完全工作在主内存中,不基于文件系统,支持事务,支持部分SQL,支持稳定的RAM。通过数据库定义语言为应用系统各自的API。具有工业应用强度。\\
FastDB & 开源 & FastDB 是一个高效率的内存数据库系统,支持事务、在线备份和系统崩溃之后的自动恢复,支持类似SQL语言并提供了C++接口,fastDB算法和结构的优化都是基于数据存放在内存中这个假设上,但物理内存较小时也可使用。\\
\bottomrule[1.5pt]
\end{tabularx}}
\vspace{\baselineskip}
\end{table}

\chapter{系统实现技术方法研究}


% \makeatother
\backmatter
\endgroup % 组结束
%%%%%%%%%% 参考文献 %%%%%%%%%%
\clearpage % 显式换页,使书签定位准确
\bibliography{references/literaturereview_reference}
\nocite{*}                                   % 若将此命令屏蔽掉,则未引用的文献不会出现在文后的参考文献中。

%%%%%%%%%% 附录 %%%%%%%%%%
%\appendix
%\include{appendix/appendix}            % 附录

\end{document}                                  % 结束全文
