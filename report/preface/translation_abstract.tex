% !Mode:: "TeX:UTF-8"


%%%%% 摘要 %%%%%
\abstractc{在线存储的磁盘的方法正成为越来越严重的问题:他们的规模不能很好的满足大型Web应用程序,对磁盘容量的改善已经远远超过了在访问延迟和带宽的改善。本文提出一种关于数据中心存储新的方法:RAMCloud,信息完全保存在DRAM和大型系统所建立的聚集成千上万的商品服务器的主存储器上。我们认为,RAMClouds可以
提供基于磁盘的系统、100-1000倍低访问延迟和吞吐量的持久的和可用的存储。低延迟和大规模的结合,提供了一种新的数据密集型应用程序。}

%%%%% 关键词 %%%%%
\keywordsc{RAMCloud,DRAM,高性能,大数据}

% %%%%% 摘要 %%%%%
% \abstractc{内存驻留数据库系统(MMDB)将数据存储在
% 其主要物理内存中并提供非常高的速度
% 对数据进行访问。传统的数据库系统主要针对
% 磁盘存储机制的具体特点来进行优化。相反的是,内存常驻系统对结构和组织数据使用不同的优化
% 从而使其可靠。
% 本文概述的主要是内存常驻的优化
% 并简要讨论了一些常驻内存系统的
% 设计和实施。}

% %%%%% 关键词 %%%%%
% \keywordsc{访问方式,应用编程
% 接口,事务提交处理,并发控制,数据聚类,
% 数据表示,主内存数据库系统(MMDB),
% 查询处理,数据恢复}

