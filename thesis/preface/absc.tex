% !Mode:: "TeX:UTF-8"
%% 中文摘要
\begin{abstractc}
现如今随着互联网的不断发展,关系型数据库已经在业界的数据存储方面占据不可动摇的地位,但是由于本身的限制,使其很难满足“易扩展”、“读写速度快”、“低成本”、“海量数据的支持”等一些条件,这些因素促使了内存数据库的发展。

内存数据库将数据完全放在内存中,这样做的好处是对数据有极高的处理速度,而传统的硬盘只用做数据库的持久化备份。由于Redis本身是一个高性能的Key-Value数据库系统,本系统便底层基于Redis和SQLite搭建,从而在Redis的基础上实现了基础SQL语句的操作,上层GUI基于PyQt4开发。主要涉及的技术包括数据的存储,数据的持久化,数据的恢复,数据的可视化等。

本系统支持基础的SQL操作,持久化时根据SQLite的特性将整个数据库保存成单个文件的形式。于此同时,拿其与MySQL5.5进行了对比,比较了两者在读写速度上的差异,并讨论了该系统相关的应用场景。

最后一部分包含了本系统和MySQL5.5性能对比的结果,并做了一定的分析,评价和总结。

% 摘要内容,小四号宋体,段前段后0磅,1.5倍间距。500字左右。每段开头空两格,标点符号占一格。中文摘要应表达毕业设计工作的核心内容,简短明了。
% 首先,摘要应当要素齐全。即一篇摘要应当包含如下要素:
% 1.目的—即从事该项研究开发的理由与背景或所涉及的主题范围;
% 2.方法—即所用的原理﹑理论﹑开发工具,关键技术解决方法等;
% 3.结果—即研究开发工作的结果﹑数据﹑效果﹑性能等;
% 4.结论—即对结果的分析﹑评价等。
% 其次,摘要应当客观﹑如实地反映论文的内容。
% 第三,采用第三人称写法。由于摘要将直接被检索类二次文献采用,脱离原文独立存在,所以摘要一律采用第三人称写法。

\keywordsc{Redis,内存数据库,SQLite}
\end{abstractc}
