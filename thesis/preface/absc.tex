% !Mode:: "TeX:UTF-8"
%% 中文摘要
\begin{abstractc}
内存数据库是最近很热门的技术,
它将数据完全放在内存中,
这样做的好处是对数据有极高的处理速度,
而传统的硬盘只用做数据库的持久化备份。
在这之中,具有代表性的Redis是一个高性能的Key-Value数据库系统,
在此基础上,本内存数据库系统结合SQLite进行开发,
从而在Redis的基础上实现了基础SQL语句的操作,
上层GUI基于PyQt4开发。主要涉及的技术包括数据的存储,
数据的持久化等。

本内存数据库应用系统在Redis原命令的基础上支持一些基础的SQL操作(Create,Select等),
同时根据Redis Server的配置,
在持久化时根据SQLite的特性将整个数据库保存成单个文件的形式。
与此同时,拿其与比较热门的关系型数据库MySQL(5.5)进行了对比,
比较了两者在读写速度上的差异。

最后一部分包含了本系统和MySQL5.5性能对比的结果,并做了一定的分析,评价和总结。


% 摘要内容,小四号宋体,段前段后0磅,1.5倍间距。500字左右。每段开头空两格,标点符号占一格。中文摘要应表达毕业设计工作的核心内容,简短明了。
% 首先,摘要应当要素齐全。即一篇摘要应当包含如下要素:
% 1.目的—即从事该项研究开发的理由与背景或所涉及的主题范围;
% 2.方法—即所用的原理﹑理论﹑开发工具,关键技术解决方法等;
% 3.结果—即研究开发工作的结果﹑数据﹑效果﹑性能等;
% 4.结论—即对结果的分析﹑评价等。
% 其次,摘要应当客观﹑如实地反映论文的内容。
% 第三,采用第三人称写法。由于摘要将直接被检索类二次文献采用,脱离原文独立存在,所以摘要一律采用第三人称写法。

\keywordsc{Redis,内存数据库,SQLite}
\end{abstractc}
