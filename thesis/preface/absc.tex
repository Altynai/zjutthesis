% !Mode:: "TeX:UTF-8"
%% 中文摘要
\begin{abstractc}
随着互联网的高速发展,各种大数据类型的应用层出不穷。低延迟I/O等高需求的条件对传统的磁盘数据库产生了很大冲击。内存数据库是最近很热门的技术,它将数据完全放在内存中,而传统的硬盘只用做数据库的持久化备份,使得应用程序对数据有极高的处理速度。

本文基于Redis及SQLite等开源数据库,实现了一个大数据的应用系统,重点对Redis底层数据结构的设计原理及方法进行了分析。使用Python语言编写了主要功能的模块,其中GUI部分基于扩展库PyQt4开发。为方便用户使用,本系统在Redis原命令的基础上实现了一些基础的SQL操作(Create,Select 等),并能够在GUI中显示相应的操作结果。在性能测试方面,重点对数据读取性能与热门的关系型数据库MySQL(5.5)进行了对比,比较了两者在读写速度上的差异。

实验结果表明,本系统实现了内存数据库的基本操作。与磁盘数据库相比,在读写性能上有较大的提高,可以满足数据挖掘大数据应用的需求。


% 摘要内容,小四号宋体,段前段后0磅,1.5倍间距。500字左右。每段开头空两格,标点符号占一格。中文摘要应表达毕业设计工作的核心内容,简短明了。
% 首先,摘要应当要素齐全。即一篇摘要应当包含如下要素:
% 1.目的—即从事该项研究开发的理由与背景或所涉及的主题范围;
% 2.方法—即所用的原理﹑理论﹑开发工具,关键技术解决方法等;
% 3.结果—即研究开发工作的结果﹑数据﹑效果﹑性能等;
% 4.结论—即对结果的分析﹑评价等。
% 其次,摘要应当客观﹑如实地反映论文的内容。
% 第三,采用第三人称写法。由于摘要将直接被检索类二次文献采用,脱离原文独立存在,所以摘要一律采用第三人称写法。

\keywordsc{Redis,内存数据库,SQLite,备份恢复}
\end{abstractc}
