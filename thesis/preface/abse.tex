% !Mode:: "TeX:UTF-8"
%% 英文摘要
\begin{abstracte}
With the high-speed development of the Internet, all kinds of applications with big data are coming out. Low latency I/O has a great impact on traditional disk database. Memory database is a very popular technology recently, it stores the data in memory entirely, which allows the application to process the data with a high processing speed, meanwhile a traditional hard drive is only used for persistent backup of the database.

Based on open source memory databases such as Redis and SQLite, this paper implements a big data application, focusing on how the underlying data structures are designed and method are analyzed in Redis. The main function modules are written by the Python language, including the GUI part based on the extension libraries PyQt4. For the convenience of users, this system not only includes the Redis original command but also implements some basic SQL operations (Create, Select, etc.), and it can display corresponding operation results in the GUI. In terms of the performance testing, the paper compared this application with the popular relational database MySQL (5.5), 
including the read/write speed difference between them.

Experimental results show that the application has realized the basic operations of the memory database. Compared with the disk database, it has larger increase in the read/write performance, which can meet the demand of big data applications such as data mining.
% Externally pressurized gas bearing has been widely used in the field of aviation,
% semiconductor, weave, and measurement apparatus because of its advantage of high accuracy,
% little friction, low heat distortion, long life-span, and no pollution.
% In this \textit{thesis}, based on the domestic and overseas researching\ldots\ldots

\keywordse{~~Redis,~~Main Memory Database,~~SQLite,~~Data Recovery}
\end{abstracte}
