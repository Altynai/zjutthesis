% !Mode:: "TeX:UTF-8"

\documentclass{zjutthesis}

\graphicspath{{figures/}}  % 定义所有的eps文件在 figures 子目录下
\begin{document}           % 开始全文

%论文题目:{中文}
\zjuttitle{基于内存数据库的大数据应用系统的设计与实现}
%作者:{中文姓名}{学号}
\zjutauthor{陈佳鹏}{20092663503}
%指导教师:{导师中文名}
\zjutmentor{陈~~~~波}
%个人信息:{毕业年份}{专业名称}
\zjutinfo{2013}{软件工程}
%学院信息:{学院中文}
\zjutcollege{计算机科学与技术学院}
%日期:{提交日期}
\zjutdate{2013年06月}

\input{preface/cover}      % 封面

\frontmatter

\pagenumbering{Roman}
% !Mode:: "TeX:UTF-8"
%% 中文摘要
\begin{abstractc}
现如今随着互联网的不断发展,关系型数据库已经在业界的数据存储方面占据不可动摇的地位,但是由于本身的限制,使其很难满足“易扩展”、“读写速度快”、“低成本”、“海量数据的支持”等一些条件,这些因素促使了内存数据库的发展。

内存数据库将数据完全放在内存中,这样做的好处是对数据有极高的处理速度,而传统的硬盘只用做数据库的持久化备份。由于Redis本身是一个高性能的Key-Value数据库系统,本系统便底层基于Redis和SQLite搭建,从而在Redis的基础上实现了基础SQL语句的操作,上层GUI基于PyQt4开发。主要涉及的技术包括数据的存储,数据的持久化,数据的恢复,数据的可视化等。

本系统支持基础的SQL操作,持久化时根据SQLite的特性将整个数据库保存成单个文件的形式。于此同时,拿其与MySQL5.5进行了对比,比较了两者在读写速度上的差异,并讨论了该系统相关的应用场景。

最后一部分包含了本系统和MySQL5.5性能对比的结果,并做了一定的分析,评价和总结。

% 摘要内容,小四号宋体,段前段后0磅,1.5倍间距。500字左右。每段开头空两格,标点符号占一格。中文摘要应表达毕业设计工作的核心内容,简短明了。
% 首先,摘要应当要素齐全。即一篇摘要应当包含如下要素:
% 1.目的—即从事该项研究开发的理由与背景或所涉及的主题范围;
% 2.方法—即所用的原理﹑理论﹑开发工具,关键技术解决方法等;
% 3.结果—即研究开发工作的结果﹑数据﹑效果﹑性能等;
% 4.结论—即对结果的分析﹑评价等。
% 其次,摘要应当客观﹑如实地反映论文的内容。
% 第三,采用第三人称写法。由于摘要将直接被检索类二次文献采用,脱离原文独立存在,所以摘要一律采用第三人称写法。

\keywordsc{Redis,内存数据库,SQLite}
\end{abstractc}
     % 中文摘要
% !Mode:: "TeX:UTF-8"
%% 英文摘要
\begin{abstracte}
With the high-speed development of the Internet, all kinds of applications with big data are coming out. Low latency I/O has a great impact on traditional disk database. Memory database is a very popular technology recently, it stores the data in memory entirely, which allows the application to process the data with a high processing speed, meanwhile a traditional hard drive is only used for persistent backup of the database.

Based on open source memory databases such as Redis and SQLite, this paper implements a big data application, focusing on how the underlying data structures are designed and method are analyzed in Redis. The main function modules are written by the Python language, including the GUI part based on the extension libraries PyQt4. For the convenience of users, this system not only includes the Redis original command but also implements some basic SQL operations (Create, Select, etc.), and it can display corresponding operation results in the GUI. In terms of the performance testing, the paper compared this application with the popular relational database MySQL (5.5), 
including the read/write speed difference between them.

Experimental results show that the application has realized the basic operations of the memory database. Compared with the disk database, it has larger increase in the read/write performance, which can meet the demand of big data applications such as data mining.
% Externally pressurized gas bearing has been widely used in the field of aviation,
% semiconductor, weave, and measurement apparatus because of its advantage of high accuracy,
% little friction, low heat distortion, long life-span, and no pollution.
% In this \textit{thesis}, based on the domestic and overseas researching\ldots\ldots

\keywordse{~~Redis,~~Main Memory Database,~~SQLite,~~Data Recovery}
\end{abstracte}
     % English Abstract

%%%%%%%%%% 目录 %%%%%%%%%%
% \defaultfont TODO format 中定义的fix
% \clearpage{\pagestyle{empty}\cleardoublepage}
% \titleformat{\chapter}{\centering\hei\sanhao\bfseries}{\chaptername}{2em}{} %设置目录两字的格式

\tableofcontents           % 中文目录
\listoffigures             % 图目录
\listoftables              % 表目录

% \defaultfont\sloppy\raggedbottom
% \titleformat{\chapter}{\centering\hei\xiaoer\bfseries}{\chaptername}{2em}{} %恢复chapter标题格式要求

%%%%%%%%%% 正文 %%%%%%%%%%
\mainmatter
% \makeatletter


%%% 第一章 绪论 %%%
% 1.1	字体标题(2级标题,小三,宋体,加粗,段前12磅,段后6磅,1.5倍行距)	1
% 1.2	图表说明	1
% 1.3	公式	2
% 1.4	本文的主要工作	2
% 1.5	本文的组织结构	2
% 1.6	本章小结	3

\chapter{绪论}
\section{课题研究背景}
1969年,埃德加•弗兰克•科德(Edgar Frank Codd)发表了一篇划时代的论文,首次提出关系数据模型的概念。但可惜的是,刊登论文的“IBM Research Report”只是IBM公司的内部刊物,因此论文反响平平。1970年,他再次在刊物《Communication of the ACM》上发表了题为“A Relational Model of Data for Large Shared Data banks”(大型共享数据库的关系模型)的论文,终于引起了大家的关注。

科德所提出的关系数据模型的概念成为了现今关系型数据库的基础。当时的关系型数据库由于硬件性能低劣、处理速度过慢而迟迟没有得到实际应用。但之后随着硬件性能的提升,加之使用简单、性能优越等优点,关系型数据库得到了广泛的应用。

但是现如今随着互联网的不断发展,各种类型的应用层出不穷,所以导致在这个云计算的时代,对技术提出了更多的需求,传统关系型数据库面临很多问题,主要体现在下面这四个方面:
\begin{enumerate}[label=(\arabic*)]
\item{低延迟的读写速度:应用快速地反应能极大地提升用户的满意度}
\item{支撑海量的数据和流量:对于搜索这样大型应用而言,需要利用PB级别的数据和能应对百万级的流量}
\item{大规模集群的管理:系统管理员希望分布式应用能更简单的部署和管理}
\item{庞大运营成本的考量:IT经理们希望在硬件成本、软件成本和人力成本能够有大幅度地降低}
\end{enumerate}

虽然关系型数据库已经在业界的数据存储方面占据不可动摇的地位,但是由于其天生的几个限制,使其很难满足下面的这几个需求:
\begin{enumerate}[label=(\arabic*)]
\item{扩展困难:当需要表与表之间Join操作时,数据库就不能很好的进行扩展}
\item{读写速度慢:当数据量达到较大规模的时候,很容易发生死锁的问题,直接导致读写速度急剧下降的问题}
\item{成本高:企业级数据库的License价格惊人,且随着系统的规模而不断上升}
\item{有限的支撑容量:现有关系型解决方案还无法支撑Google这样海量的数据存储}
\end{enumerate}

而因此思想诞生的内存数据库(MMDB)对数据访问读取有着很高的效率,对于一些需要
在严格要求的时间段内完成事务请求
的实时应用系统,和需要支持大量并发访问的高性能事务处理平台来而言,它都是一个很不错的选择。

\section{研究与应用现状}
国内外相关人员经过长年的研究,设计并实现了多种内存数据库的
模型及商用或开源的产品。商用MMDB的代表产品有AltiBase、Timesten,SAP HANA等,开源
产品主要有FastDB、SQLite、Redis等,下文对一些数据库做简单的介绍:

\subsection{Oracle TimesTen}
Oracle TimesTen内存数据库是一个功能全面的关系型内存数据库,旨在通过在应用层的运行,加速处理响应时间和关键任务型应用所需的高吞吐量。从某种角度上来看,TimesTen也是一种Cache机制,是磁盘数据库的‘Cache’,通过物理内存中的数据存储区的直接操作,减少了到磁盘间的I/O交互。TimesTen中的这个Ten据说就是指速度能达到基于磁盘的RDBMS10倍。

\subsection{SAP HANA}
SAP HANA是一款面向数据源的、灵活、多用途的内存应用设备,整合了基于硬件优化的SAP软件模块,通过SAP主要硬件合作伙伴提供给客户。SAP HANA提供灵活、节约、高效、实时的方法管理海量数据。利用HANA,企业可以不必运行多个数据仓库、运营和分析系统,从而削减相关的硬件和维护成本。SAPHANA将在内存技术基础上,为新的创新应用程序奠定技术基础,支持更高效的业务应用程序,如:计划、预测、运营绩效和模拟解决方案。

\subsection{Redis}
Redis是一个高性能的key-value存储系统。和Memcached类似,它支持存储的value类型相对更多,包括string(字符串)、list(链表)、set(集合)和zset(有序集合)。这些数据类型都支持push/pop、add/remove及取交集并集和差集及更丰富的操作,而且这些操作都是原子性的。在此基础上,redis支持各种不同方式的排序。与memcached一样,为了保证效率,数据都是缓存在内存中。区别的是redis会周期性的把更新的数据写入磁盘或者把修改操作写入追加的记录文件,并且在此基础上实现了master-slave(主从)同步。

\section{本文主要工作}
(1)了解Redis 的工作原理,包括内部数据结构(动态字符串、双端列表、字典、跳
跃表),内存映射数据结构(整数集合、压缩列表),数据类型(对象处理机制、
字符串、哈希表、列表、集合、有序集),功能的实现(事务、订阅与发布、慢查
询日志),内部运作机制(数据库、rdb、aof、事件、服务器与客户端)

(2)设计开发图形化的性能测试工具:能反应出内存数据库和磁盘
数据库之间性能的差距

\section{本文的组织结构}
本文共分为八章,以内存数据库为背景,研究讨论了ExtJS+DWR+Spring+Hibernate的Web应用架构,以及在每层所采用的开源框架,详细阐述了如何利用该框架技术对系统的模块进行设计与实现,各章内容如下:

第一章,介绍了课题研究的背景,国内外相关领域的研究及应用,课题研究的主要任务和本文的主要工作。

第二章,详细介绍了系统开发的方法与技术,为系统的后续开发做准备。

第三章,重点介绍了教学改革与建设项目中期检查系统的需求。

第四章,具体介绍了教学改革与建设项目中期检查系统的概要设计。其内容主要包括系统主要功能的业务流程及详细描述和数据库设计。

第五章,详细介绍了教学改革与建设项目中期检查系统的详细设计。其内容包括开发规范的确定、系统所采用框架的整合设计、功能模块的详细设计和系统性能要求的详细设计。

第六章,着重阐述了教学改革与建设项目中期检查系统的具体实现,针对第五章提出的详细设计要求,在本章给出系统的技术实现,具体包括系统框架整合实现、功能性能要求实现和系统功能模块实现。

第七章,系统测试与系统使用说明。

第八章,对系统开发进行总结并提出下一步工作。


%%% 第二章 方法与技术 %%%
% 2.1	Spring框架简介	4
% 2.1.1	Spring的控制反转(IoC)(3级标题,四号,宋体,加粗,段前6磅,1.5倍行距,建议不使用四级或更高级别目录、标题)	4
% 2.1.2	面向切面编程(AOP)	4
% 2.1	Hibernate框架	4
% 2.2	AJAX	5
% 2.3	Ext框架	5
% 2.4	DWR技术	5
% 2.5	开发环境	5
% 2.5.1	服务器端环境要求	5
% 2.5.2	客户端环境要求	5
% 2.6	主要开发语言	6
% 2.7	开发原则	6
% 2.8	本章小结	6

\chapter{方法与技术}
本系统采用了后端Redis配合SQLite,前端GUI使用PyQT4的架构,实现了支持一些基础SQL语句的内存数据库系统,本章将对上述知识进行简要的阐述。

\section{Redis}
Redis是一个开源的使用ANSI C语言编写、支持网络、可基于内存亦可持久化的日志型、Key-Value数据库,并提供多种语言的API。作为Key-value型数据库,Redis也提供了键(Key)和键值(Value)的映射关系。但是,除了常规的数值或字符串,Redis的键值还可以是以下形式之一:

Lists(列表)

Sets(集合)

Sorted sets(有序集合)

Hashes(哈希表)

键值的数据类型决定了该键值支持的操作。Redis支持诸如列表、集合或有序集合的交集、并集、差集等高级原子操作;同时,如果键值的类型是普通数字,Redis则提供自增等原子操作。通常,Redis将数据存储于内存中,或被配置为使用虚拟内存。通过两种方式可以实现数据持久化:使用快照的方式,将内存中的数据不断写入磁盘;或使用类似MySQL的日志方式,记录每次更新的日志。前者性能较高,但是可能会引起一定程度的数据丢失;后者相反。

相比需要依赖磁盘记录每个更新的数据库,基于内存的特性无疑给Redis带来了非常优秀的性能。读写操作之间没有显著的性能差异,如果Redis将数据只存储于内存中。


\section{SQLite}
SQLite是遵守ACID的关系数据库管理系统,它包含在一个相对小的C库中。它是D.RichardHipp创建的公有领域项目。
不像常见的客户端/服务器结构范例,SQLite引擎不是个程序与之通信的独立进程,而是连接到程序中成为它的一个主要部分。所以主要的通信协议是在编程语言内的直接API调用。这在消耗总量、延迟时间和整体简单性上有积极的作用。整个数据库(定义、表、索引和数据本身)都在宿主主机上存储在一个单一的文件中。它的简单的设计是通过在开始一个事务的时候锁定整个数据文件而完成的。

\section{Python}
Python是个成功的脚本语言。它最初由 Guido van Rossum 开发,在1991年第一次发布。Python由ABC和Haskell语言所启发。Python是一个高级的、通用的、跨平台、解释型的语言。一些人更倾向于称之为动态语言。它很易学,Python是一种简约的语言。它的最明显的一个特征是,不使用分号或括号,Python使用缩进。最近的版本是2.7(3.2),2011年二月发布。现在,Python由来自世界各地的庞大的志愿者维护。Python是那些想要学习编程的人的理想的入门语言。
Python支持多种编程模式,它不强制程序员使用特定的模式。Python支持面向对象和面向过程编程,也有限的支持函数式编程。

\section{PyQT4}
PyQT是一个生成图形应用程序的工具包。是python语言和成功的Qt库的绑定。Qt库是这个世界上最强大的库之一,PyQT作为一组python的模块来实现。它包含了超过300个类,将近6000个函数和方法。它是一个多平台的工具包,可以在所有的主流操作系统上运行,包括Unic,Windows和Mac。PyQT采用双协议,开发者可以在GPL和商业授权中选择。以前的版本中,GPL版本只存在于Unic上。从PyQT4开始,GPL协议支持所有的平台。QtCore模块包含了核心的非图形功能,这个模块被用来实现时间,文件和目录,不同的数据格式,流,互联网地址,mime类型,线程或进程等等。QtGui模块包含了图形组件和类的描述,包括例如按钮,窗口,状态栏,滑块,位图,颜色,字体等等。

QtCore模块包含了核心的非图形功能,这个模块被用来实现时间,文件和目录,不同的数据格式,流,互联网地址,mime类型,线程或进程等等。

QtGui模块包含了图形组件和类的描述,包括例如按钮,窗口,状态栏,滑块,位图,颜色,字体等等。

QtNetwork模块包含了网络编程所需的类,这些类可以用来实现TCP/IP和UDP的客户端/服务器程序,使得网络编程更加简单更加可移植。

QtXml模块提供了处理xml文件的类,这个模块包含了SAX和DOM APIs的实现。

QtSvg模块包含了显示SVG文件目录的类。Scalable Vector Graphics (SVG)是一种利用XML来描述二维图形和图形应用的的语言。

QtOpenGL模块用来通过OpenGL库来渲染3D和2D图形,这个模块实现了Qt GUI库和OpenGL库的无缝集合。

QtSql模块提供了处理数据库的类。

\section{本章小结}
本章以本系统开发的相关理论及技术为基础,介绍系统开发过程中需要了解和掌握的方法和技术,详细阐述了Redis,SQLite,PyQT4d等相关技术。


\chapter{需求分析}
% 3.1	系统简介	7
% 3.1.1	项目类别和申报类型	7
% 3.1.2	系统使用对象	7
% 3.1.3	功能概述	7
% 3.2	系统的整体框架	7
% 3.3	本章小结	8

\section{系统简介}
由于Redis本身Key-Value的特性,它可以在很多Web应用场景上发挥巨大的功效,下文列举了几个常见的应用场景:
\begin{enumerate}[label=(\arabic*)]
\item{在主页中显示最新的项目列表。
由于Redis使用的是常驻内存的缓存,速度非常快。LPUSH用来插入一个内容ID,作为关键字存储在列表头部。LTRIM用来限制列表中的项目数最多为5000。如果用户需要的检索的数据量超越这个缓存容量,这时才需要把请求发送到数据库。}

\item{排行榜及相关问题。
排行榜按照得分进行排序,ZADD命令可以直接实现这个功能,而ZREVRANGE命令可以用来按照得分来获取前N名的用户,ZRANK可以用来获取用户排名}

\item{按照用户投票和时间排序。
这就像Reddit的排行榜,得分会随着时间变化。LPUSH和LTRIM命令结合运用,把文章添加到一个列表中。一项后台任务用来获取列表,并重新计算列表的排序,ZADD命令用来按照新的顺序填充生成列表。列表可以实现非常快速的检索,即使是负载很重的站点。}

\item{计数。进行各种数据统计的用途是非常广泛的,比如想知道什么时候封锁一个IP地址。INCRBY命令让这些变得很容易,通过原子递增保持计数;GETSET用来重置计数器;过期属性用来确认一个关键字什么时候应该删除。}

\item{队列。在当前的编程中队列随处可见。除了push和pop类型的命令之外,Redis还有阻塞队列的命令,能够让一个程序在执行时被另一个程序添加到队列。你也可以做些更有趣的事情,比如一个旋转更新的RSS feed队列。}

\item{缓存。
Redis缓存使用的方式与memcached相同。
网络应用不能无休止地进行模型的战争,看看这些Redis的原语命令,尽管简单但功能强大,把它们加以组合,所能完成的就更无法想象。当然,你可以专门编写代码来完成所有这些操作,但Redis实现起来显然更为轻松。}
\end{enumerate}

但是,Redis并非是传统的关系型数据库,无法支持SQL语句解析,所以本系统在这基础上配合采用了SQLite,同时为Redis新增加了一个基于SQLite的“sql”命令。至此,可以通过sql命令来进行数据库的表操作,表中的所有数据完全存储在内存中,此外,根据Redis.conf的配置,可以设置表中数据库每次保存到硬盘的间隔,这样做可以保证数据的正确性,防止出现可能的断电宕机使数据丢失的情况。

\section{系统框架}
整个系统的框架主要分两部分:Server(服务器端)和Client(客户端)。
%%% TODO 架构图 %%%
\subsection{服务器端}
这里的服务器端就是Redis本身的Server, 除了维持服务器状态之外, 最重要的就是将Redis的各个功能模块组合起来。
简单而言,服务器端就是整个系统的“心脏”,它维护着各个模块,保证每个模块稳定健康的运行。

\subsection{客户端}
客户端的作用就是与服务器端进行交互,通过指令来操作控制服务器端。两者之间的交互过程如下:客户端向服务器发送命令,服务器接受命令然后将命令传给命令执行器,执行器执行给定命令的实现函数,执行完成之后,将结果保存在缓存,最后回传给客户端。同时,服务器端可以接受来自多个客户端发来的命令请求。

\section{本章小结}
本章对内存数据库系统需求进行了充分的分析,明确了系统框架、大致功能结构等问题,为后续系统设计打下了基础。


%%% 第四章 项目中期检查系统概要设计	%%% 
\chapter{概要设计}
% 4.1	系统业务流程	9
% 4.1.1	业务流程	9
% 4.2	系统功能结构	10
% 4.3	系统架构设计	11
% 4.4	系统数据库设计	11
% 4.5	本章小结	15
\section{系统业务流程}
\subsection{业务流程}
本系统的大体业务流程(%%% TODO 业务流程图
):客户端向服务器发送命令,服务器接受命令然后将命令传给命令执行器,执行完成之后,将结果保存在缓存,最后回传给客户端。

服务器启动之后,它便在6379端口监听来自客户端的请求,此时客户端可以连接到某个服务器,然后发送指令请求。如果客户端连接对应的服务器未启动,那客户端发送的命令的请求都会返回错误。

\section{系统功能结构}
本系统功能主要分两部分:Redis元命令部分,关系型数据库表结构部分。系统的功能结构如%%% TODO %%%
所示。

\section{系统架构设计}
系统的架构设计如%%% TODO %%%
所示。

\section{本章小结}
本章主要对内存数据库系统进行了概要设计。首先,对本系统进行了详细的阐述并进行系统架构设计。然后,确定了系统的总体功能结构,概要描述了各个功能模块的详细要求。


%%% 第五章 项目中期检查系统详细设计	 %%%
% 5.1	项目开发规范	16
% 5.1.1	系统目录规划	16
% 5.1.2	命名规则	17
% 5.2	系统功能模块详细设计	17
% 5.3	系统性能优化设计	19
% 5.4	本章小结	19
\chapter{系统详细设计}
\section{项目开发规范}
开发规范在项目的开发过程中具有非常重要的作用,良好的开发规范可以提高软件开发质量,降低开发周期,增强代码的可重用性和易读性,使软件便于维护。

% \makeatother
\backmatter

%%%%%%%%%% 参考文献 %%%%%%%%%%
\bibliography{references/reference}
\nocite{*}                                   % 若将此命令屏蔽掉,则未引用的文献不会出现在文后的参考文献中。

%%%%%%%%%% 致谢 %%%%%%%%%%
% !Mode:: "TeX:UTF-8"

%\titlecontents{chapter}[2em]{\vspace{.5\baselineskip}\wuhao\hei}
%{\prechaptername\CJKnumber{\thecontentslabel}\postchaptername\qquad}{}
%{\hspace{.5em}\titlerule*[10pt]{$\cdot$}\wuhao\contentspage}
\chapter{\heiti\bfseries{致谢}}
由于自己本身的能力有限,虽然此次毕业设计已经基本完成,但是其中还有很多的不足和有待改进之处。

在这里特别感谢我的导师陈波老师在这整个过程中给予我的悉心的指导,以及一起努力奋斗的同学们的支持。感谢同组的杨道峰同学,同寝室的几位室友,感谢他们一直的支持和鼓励。感谢大学以来的老师,教会了我们很多重要的基础知识。最后感谢浙江工业大学四年来对我的大力培养。
            % 致谢

%%%%%%%%%% 附录 %%%%%%%%%%
\appendix
\include{appendix/appendix}            % 附录

% 以下注释内容需放在第一个附录tex文件的头部,放在主文件里会造成“附录”两字单独成页。
%\setlength{\parskip}{18pt}
%\chapter*{\centering\hei\xiaoer{附\qquad 录}}
%\setlength{\parskip}{18pt}
%\setlength{\parskip}{0pt}

\end{document}                                  % 结束全文
