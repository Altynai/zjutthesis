% !Mode:: "TeX:UTF-8"

\chapter{开发生命周期}
开发框架的重点在一个开发团队中可以明确定义角色及其相互作用的结构上。三种角色描述如下。这些角色的相互作用是整个应用项目成功的关键。(a)前端的开发人员专注于JSPs,Action/ActionForm类和外部Web服务。(b)服务的开发者专注于开发应用的服务和整合这些服务中的不同部分。(c)项目集成者主要专注开发的集成文件,如DAOs或消费型Web服务。

发展中的一个基本问题是在其依赖组件没有准备好或不可用时如何开发和集成的代码。开发框架通过以声明式注入“模拟对象”这种结构来解决这个问题,并在开发生命周期的过程中用实际对象取代模拟对象。由于我们的应用是通过不同的配置集进行配置的使这成为了可能。该框架使团队能测试开发过程中的一个组成部分。这使编写和运行JUnit测试成为了可能。框架专注于测试应用服务和他们的依赖性。应用部署在一个单一的Enterprise Archive(EAR)文件中。Ant脚本生成此EAR文件,并可以手动运行或定期调度。建议在创建EAR之前运行所有的JUnit测试。
