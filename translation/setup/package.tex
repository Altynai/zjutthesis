% !Mode:: "TeX:UTF-8"
% !Tex Program = xelatex
% Author: Unlucky(http://blog.thebeyond.name)
%         Myautsai(http://mckelv.in)

%%%%%%%%%% Package %%%%%%%%%%%%
\usepackage{graphicx}                       % 支持插图处理
\usepackage[a4paper,
              text={148true mm,223true mm},
			  top=37true mm,
			  bottom=37true mm,
			  left=31true mm,
			  right=31true mm,
			  head=7true mm,
			  headsep=2.5true mm,
			  foot=7true mm
			]{geometry}                      % 支持版面尺寸设置
\usepackage{titlesec}                       % 控制标题的宏包
\usepackage{titletoc}                       % 控制目录的宏包
\usepackage{fancyhdr}                       % fancyhdr宏包 支持页眉和页脚的相关定义
%\usepackage[UTF8]{ctex}
\usepackage[CJKnumber]{xeCJK}                 %xeCJK中文支持,并提供将阿拉伯数字转换成中文数字的命令
\usepackage{color}                          % 支持彩色
\usepackage{amsmath}                        % AMSLaTeX宏包 用来排出更加漂亮的公式
\usepackage{amssymb}                        % 数学符号生成命令
\usepackage[below]{placeins}                %允许上一个section的浮动图形出现在下一个section的开始部分,还提供\FloatBarrier命令,使所有未处理的浮动图形立即被处理
\usepackage{flafter}                        % 使得所有浮动体不能被放置在其浮动环境之前,以免浮动体在引述它的文本之前出现.
\usepackage{multirow}                       % 使用Multirow宏包,使得表格可以合并多个row格
\usepackage{booktabs}                       % 表格,横的粗线;\specialrule{1pt}{0pt}{0pt}
\usepackage{longtable}                      % 支持跨页的表格。
\usepackage{tabularx}                       % 自动设置表格的列宽
\usepackage{subfigure}                      % 支持子图 %centerlast 设置最后一行是否居中
\usepackage[subfigure]{ccaption}            % 支持子图的中文标题
\usepackage[sort&compress,numbers]{natbib}  % 支持引用缩写的宏包
\usepackage{enumitem}                       % 使用enumitem宏包,改变列表项的格式
\usepackage{calc}                           % 长度可以用+ - * / 进行计算
\usepackage{txfonts}                        % 字体宏包
\usepackage{bm}                             % 处理数学公式中的黑斜体的宏包
\usepackage[amsmath,thmmarks,hyperref]{ntheorem}  % 定理类环境宏包,其中 amsmath 选项用来兼容 AMS LaTeX 的宏包
%\usepackage{CJKnumb}                        % 提供将阿拉伯数字转换成中文数字的命令
\usepackage{indentfirst}                    % 首行缩进宏包
%\usepackage{CJKutf8}                        % 用在UTF8编码环境下,它可以自动调用CJK,同时针对UTF8编码作了设置。
%\usepackage{hypbmsec}                      % 用来控制书签中标题显示内容
\usepackage{CJKfntef}

\usepackage{times}
\usepackage{fontspec,xunicode,xltxtra}      % XeLaTeX相关字体字库

%如果您的pdf制作中文书签有乱码使用如下命令,就可以解决了
\usepackage[xetex, unicode,
              pdfstartview=FitH,
              CJKbookmarks=true,
              bookmarksnumbered=true,
              bookmarksopen=true,
              colorlinks=true,
			  citecolor=black,
              linkcolor=black,
              anchorcolor=black,
              urlcolor=black,
              breaklinks=true
            ]{hyperref}
